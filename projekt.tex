\documentclass[12pt,oneside,slovak,a4paper]{article}

\usepackage[slovak]{babel}
\usepackage[utf8]{inputenc}
\usepackage{amsmath}
\usepackage{amsfonts}
\usepackage{amssymb}
\usepackage{graphicx}
\usepackage{cite}
\usepackage[IL2]{fontenc} % lepšia sadzba písmena Ľ než v T1
\usepackage{pdfpages}
\usepackage{url} % príkaz \url na formátovanie URL
\usepackage{hyperref} % odkazy v texte budú aktívne (pri niektorých triedach dokumentov spôsobuje posun textu)
\usepackage[left=2cm,right=2cm,top=2cm,bottom=2cm]{geometry}
\usepackage{float}
\usepackage[normalem]{ulem}
\useunder{\uline}{\ul}{}
\usepackage{titling}


%\title{   
%Voľne šíriteľné nástroje na obnovu zmazaných súborov}
%\author{Marek Čederle\\[2pt]
%	{\small Slovenská technická univerzita v Bratislave}\\
%	{\small Fakulta informatiky a informačných technológií}\\
%	{\small \texttt{xcederlem@stuba.sk}}
%	}
%\date{\small \today}
%
\begin{document}

%\vspace{50pt}
%\maketitle
%\vspace*{\fill}
%\pagebreak

\begin{titlepage}
	\centering
    {\Large Slovenská technická univerzita v Bratislave\par}
    {\Large Fakulta informatiky a informačných technológií\par}
	\vspace{7cm}
	{\huge\bfseries Voľne šíriteľné nástroje na obnovu zmazaných súborov\par}
	\vspace{0.5cm}
    {\Large \textsc{Princípy informačnej bezpečnosti}\par}
    \vspace{1cm}
	{\Large\itshape Marek Čederle\par}
    {\small\texttt{xcederlem@stuba.sk}\par}
	\vfill

% Bottom of the page
	{\large \today\par}
\end{titlepage}


\tableofcontents
\vspace*{\fill}

\section{Špecifikácia projektu / Úvod}
V mojom projekte sa budem venovať analýze súborových systémov pre operačné systémy Windows a GNU+Linux. Bude sa jednať o NTFS a ext ale spomeniem aj dodnes veľmi používaný FAT32 a exFAT ktoré sa používajú na prenosných médiách. Každý súborový systém by som chcel rozviesť s tým, že uvediem jeho výhody a nevýhody prípadné porovnanie s ďalšími spomenutými súborovými systémami. Taktiež sa budem zaoberať analýzou nástroju na obnovu zmazaných súborov s názvom testdisk. Vysvetlím prečo som si zrovna vybral tento nástroj. S týmto nástrojom budem následne experimentovať. Experimenty budú spočívať v tom že si naformátujem disk na daný filesystém a vytvorím na ňom nejaké partície. Následne naň uložím rôzne typi súborov. Budú sa tam nachádzať fotky, textové súbory, spustiteľné súbory, atď. Potom vymažem nejaké súbory ale na disk nič nezapíšem aby sa nezačali prepisovať dané miesta na disku iným súborom. Následne vyskúšam nástroj na obnovu zmazaných súborov (testdisk) či zvládne tieto súbory obnoviť. Ďalší experiment bude spočívať v zmazaní celej partície a jej následnej obnove týmto nástrojom.
\subsection{Progress report č.1}

\subsection{Progress report č.2}

%\section{Úvod}
%Tu môžete predstaviť\cite{TEST} svoju tému a poskytnúť stručný prehľad toho, o čom budete vo svojom projekte diskutovať.

\section{Analýza súborových systémov}
Test\cite{TEST}
\subsection{NTFS}
\subsection{ext}
\subsection{FAT}
\subsubsection{FAT32}
\subsubsection{exFAT}


\section{Analýza nástroja na obnovu údajov - testdisk}

\section{Experimentovanie s nástrojom testdisk}
\subsection{Zmazanie a obnova súborov}
\subsection{Zmazanie a obnova partície}

\section{Výsledky experimentov}

\section{Záver}


\bibliography{literatura}
\bibliographystyle{alpha}
\end{document}